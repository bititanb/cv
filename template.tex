%!TEX TS-program = xelatex
%!TEX encoding = UTF-8 Unicode

\documentclass[$fontsize$, a4paper]{article}

% LAYOUT
%--------------------------------
% Margins
\usepackage{geometry}
\geometry{$geometry$}

% Do not indent paragraphs
\setlength\parindent{0in}

% Enable multicolumns
\usepackage{multicol}
% \setlength{\columnsep}{-3.5cm}
% \setlength{\columnsep}{}

% Uncomment to suppress page numbers
\pagenumbering{gobble}

% LANGUAGE
%--------------------------------
% Set the main language
$if(lang)$
\usepackage{polyglossia}
\setmainlanguage{$lang$}
$endif$

% TYPOGRAPHY
%--------------------------------
\usepackage{fontspec}
\usepackage{xunicode}
\usepackage{xltxtra}
% converts LaTeX specials (quotes, dashes etc.) to Unicode
\defaultfontfeatures{LetterSpace=2, Mapping=tex-text}
\setromanfont [Ligatures={Common}, Numbers={OldStyle}]{$mainfont$}
% Cool ampersand
\newcommand{\amper}{{\fontspec[Scale=.95]{$mainfont$}\selectfont\itshape\&}}

% MARGIN NOTES
%--------------------------------
% \usepackage{marginnote}
% \newcommand{\note}[1]{\marginnote{\scriptsize #1}}
% \renewcommand*{\raggedleftmarginnote}{}
% \setlength{\marginparsep}{7pt}
% \reversemarginpar

% HEADINGS
%--------------------------------
\usepackage{sectsty}
\usepackage[normalem]{ulem}
\sectionfont{\rmfamily\mdseries}
\subsectionfont{\rmfamily\mdseries\scshape\normalsize}
\subsubsectionfont{\rmfamily\bfseries\upshape\normalsize}

% PDF SETUP
%--------------------------------
\usepackage{hyperref}
\hypersetup
{
  pdfauthor={$name$},
  pdfsubject={$name$'s CV},
  pdftitle={$name$'s CV},
  colorlinks, breaklinks, xetex, bookmarks,
  filecolor=black,
  urlcolor=[rgb]{0.117,0.682,0.858},
  linkcolor=[rgb]{0.117,0.682,0.858},
  linkcolor=[rgb]{0.117,0.682,0.858},
  citecolor=[rgb]{0.117,0.682,0.858}
}

% CUSTOM CONFIG
\usepackage{setspace}

\usepackage{stackengine,tikz}

% DOCUMENT
%--------------------------------
% \renewcommand{\baselinestretch}{.1}
% \setlength{\parindent}{4em}
% \setlength{\parskip}{4em}
% \onehalfspacing
\begin{document}

% {\setstretch{0.1} $name$}\\
\begin{spacing}{0.3}
{\LARGE $name$}\\
\end{spacing}

% \vspace{-20pt}

{\setlength\multicolsep{0pt}%
\begin{multicols}{2}

% $for(address)$
% $address$\\
% $endfor$

% \vspace{-10pt}

% $for(urls)$
% \href{http://$urls$}{$urls$}\\
% $endfor$

% \begin{spacing}{1}
%   \-\hspace{0.3cm}{\large 25 лет}\\
% \end{spacing}

\begin{spacing}{0.7}
  % \-\hspace{0.3cm}
  {\large DevOps}\hspace{1.55cm}{25 лет}\\
\end{spacing}

\columnbreak

\begin{flushright}
  % \begin{spacing}{0}
    \href{mailto:$email$}{$email$}\\
    $phone$\\
  % \end{spacing}
\end{flushright}

\end{multicols}
}

\vspace{30pt}

У меня нет образования (неоконченный колледж электроники) и сомнительный стаж (8 месяцев сисадмином).\\
Но, самообучаясь, за последние пару лет я получил достаточный опыт в общей автоматизации, в CI/CD, немного — в кодинге, плюс хороший общий технический бэкграунд, и мне есть что показать:\\

Демо развертывания CI/CD на локальной машине в несколько команд:\\
\url{https://bititanb.github.io/CI-CD-pipeline}\\
\hspace*{\fill}{\small Kubernetes/Docker, Ansible, Jenkins, Zabbix, ELK, Vagrant}\\
% \tikz\node[inner sep=0pt, opacity=0.3]{\small Kubernetes/Docker, Ansible, Jenkins, Zabbix, ELK, Vagrant/Packer};

Демон на C++/C (вылизанные 1100 строк + 600 строк тестов):\\


\subsection*{Areas of Interest}
$if(skills)$
\begin{itemize}
    $for(skills)$
      \item $skills$
    $endfor$
\end{itemize}
$endif$

% \vfill

\vspace{25pt}

\section*{Previous Experience}
\noindent
$for(experience)$
% \note{$experience.years$}\textsc{$experience.employer$}\\
\emph{$experience.job$}\\
$experience.city$\\[.2cm]
$endfor$

$if(education)$
\section*{Education}
\noindent
$for(education)$
% \note{$education.year$}\textbf{$education.subject$}$if(education.degree)$, $education.degree$$endif$\\
\emph{$education.institute$}$if(education.city)$, $education.city$$endif$\\[.2cm]
$endfor$
$endif$

$if(languages)$
\section*{Languages}
$for(languages)$
\emph{$languages.language$} ($languages.proficiency$)\\
$endfor$
$endif$

\end{document}
