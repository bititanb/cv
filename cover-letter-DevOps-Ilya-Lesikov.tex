\documentclass[12pt, a4paper]{article}
\usepackage{geometry}
\geometry{a4paper, left=25mm, right=25mm, top=25mm, bottom=30mm}

\usepackage{fontspec}
\defaultfontfeatures{Renderer=Basic,Ligatures={TeX}, Numbers=Lining}
\setmainfont{Georgia}
\setsansfont[FakeStretch=1.05]{Carlito}
\setmonofont[Scale=0.95, LetterSpace=-1.5, FakeStretch=0.90]{SauceCodePro Nerd Font}
\usepackage[english,russian]{babel}

\usepackage[activate={true,nocompatibility},final,tracking=true,factor=1200,stretch=20,shrink=20]{microtype}

% disable hyphenation
\usepackage[]{hyphenat}
% arithmetic operations in latex
\usepackage{calc}
% section titles configuration
\usepackage{titlesec}
% colorize text (advanced)
\usepackage{xcolor}
% multiline comments
\usepackage{verbatim}

\usepackage{setspace}

% PDF SETUP
\usepackage{hyperref}
\hypersetup {
  pdfauthor={Илья Лесиков},
  pdfsubject={Сопровод. письмо (Devops)},
  pdftitle={Сопровод. письмо (DevOps) — Илья Лесиков},
  colorlinks, breaklinks,
  filecolor=black,
  urlcolor=[RGB]{0,0,208},
  linkcolor=[RGB]{0,0,208},
  citecolor=[RGB]{0,0,208},
}

% CUSTOM COMMANDS
\newcommand{\Delimitline}{
  \vspace{-1ex}
  \noindent\makebox[\linewidth]{\rule{\DelimitlineLength}{0.12ex}} 
  \vspace{3ex}
}

\newcommand\rurl[1]{%
  \-\hspace{0.5em}
  \href{http://#1}{\nolinkurl{#1}}%
}

\newcommand\Eng[1]{%
  \foreignlanguage{english}{#1}%
}

\let\bf\textbf

\let\it\textit

\newcommand{\forceindent}{\leavevmode{\parindent=1.5em\indent}}

%% FORMATTING

\setlength{\parskip}{2ex}
% Do not indent paragraphs
\setlength{\parindent}{0ex}
% suppress page numbers
\pagenumbering{gobble}

% space between lines
\renewcommand{\baselinestretch}{1.1}

%% DOCUMENT
\begin{document}

% default font family
\sffamily

% spacing between words
\fontdimen2\font=0.2em

% maths
\newlength{\DelimitlineLength}
\setlength{\DelimitlineLength}{\textwidth+1em}

%% COLORS
\pagecolor[RGB]{245,245,245}

{\rmfamily {\Large Илья\,Лесиков, DevOps}}

\Delimitline

\forceindent У меня нет образования (неоконченный колледж электроники) и сомнительный стаж.
Но, за последний год, самообучаясь дома, я получил хороший опыт и в общей автоматизации, и в \Eng{CI/CD}. 
Получил некоторый опыт и в кодинге, как и солидный общий технический бэкграунд.

\forceindent Как подтверждение навыков — демонстрация полностью автоматизированного развертывания \Eng{CI/CD} и приложения с ним
на локальной машине (с \Eng{Kubernetes, Ansible, Jenkins, ELK, Zabbix, Vagrant/Packer}), которое я делал в течение этого года:
\vspace{0.8ex}\\
  \rurl{bititanb.github.io/CI-CD-pipeline}

Больше ссылок на GitHub и информации обо мне в резюме.

\end{document}
