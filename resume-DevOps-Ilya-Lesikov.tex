\documentclass[11pt, a4paper]{article}

%%%%%%%%%%%%%%%%%%%%%%%%%%%%%%%%%%%%%%%%%%%%%%%%
% PACKAGES CONFIG
%%%%%%%%%%%%%%%%%%%%%%%%%%%%%%%%%%%%%%%%%%%%%%%%

\usepackage{geometry}
\geometry{a4paper, left=15mm, right=15mm, top=15mm, bottom=30mm}

\usepackage{fontspec}
\defaultfontfeatures{Renderer=Basic,Ligatures={TeX}}
\setmainfont{Georgia}
\newfontfamily{\rmspacedfamily}[LetterSpace=4]{Georgia}
\setsansfont[FakeStretch=1.05]{Carlito}
\setmonofont[Scale=0.95, LetterSpace=-1.5, FakeStretch=0.90]{SauceCodePro Nerd Font}

\usepackage[main=english,russian]{babel}

\usepackage[activate={true,nocompatibility},final,tracking=true,factor=1200,stretch=20,shrink=20]{microtype}

% awesome icons
\usepackage{fontawesome}

% for strikethrough
\usepackage{ulem}

% arithmetic operations in latex
\usepackage{calc}

% section titles configuration
\usepackage{titlesec}

% colorize text (advanced), for links and stuff
\usepackage{xcolor}

% multicolumns
\usepackage{multicol}

% multiline comments
\usepackage{verbatim}

% configure top/bottom padding for itemize lists
\usepackage{enumitem}

% reversed enumerated lists
\usepackage{etaremune}

\usepackage{setspace}

% customized page numbers
\usepackage{fancyhdr}
% get number of last page for footer
\usepackage{lastpage}
% activate fancyhdr
\pagestyle{fancy}
% remove header rule
\renewcommand{\headrulewidth}{0pt}
% remove header contents
\fancyhead{}
% page numbering format
\cfoot{\vspace{1ex}Page \thepage\ of \pageref*{LastPage}}

% PDF setup
\usepackage{hyperref}
\hypersetup {
  pdfauthor  = {Ilya Lesikov},
  pdfsubject = {Resume (DevOps/SRE)},
  pdftitle   = {Resume (DevOps/SRE) — Ilya Lesikov},
  colorlinks,
  breaklinks,
  filecolor  = black,
  urlcolor   = [RGB]{0,0,208},
  linkcolor  = [RGB]{0,0,208},
  citecolor  = [RGB]{0,0,208},
}

%%%%%%%%%%%%%%%%%%%%%%%%%%%%%%%%%%%%%%%%%%%%%%%%
% CUSTOM COMMANDS
%%%%%%%%%%%%%%%%%%%%%%%%%%%%%%%%%%%%%%%%%%%%%%%%

\newcommand{\Delimitline}{
  \vspace{-2ex}
  \noindent\makebox[\linewidth]{\rule{\DelimitlineLength}{0.12ex}} }

\newcommand\rurl[1]{%
  \-\hspace{0.5em}
  \href{http://#1}{\nolinkurl{#1}}%
}

\newcommand{\forceindent}{\leavevmode{\parindent=1em\indent}}

%%%%%%%%%%%%%%%%%%%%%%%%%%%%%%%%%%%%%%%%%%%%%%%%
% FORMATTING
%%%%%%%%%%%%%%%%%%%%%%%%%%%%%%%%%%%%%%%%%%%%%%%%

% subsections spacings
\titlespacing*{\subsection}{1.3em}{2ex}{0ex}
% subsections format
\titleformat{\subsection}{\rmspacedfamily\mdseries\Large}{\thesubsection}{0.7em}{}

\setlength{\parskip}{1ex}
% do not indent paragraphs
\setlength{\parindent}{0ex}

% no border between tabular cells
\setlength{\arrayrulewidth}{0pt}
% no paddings between columns in tabular
\setlength{\tabcolsep}{0em}

% itemize lists
\setlist[itemize]{itemsep=0ex, topsep=0ex, leftmargin=2em}

% bulletpoints
\renewcommand\labelitemi{\textbullet}

%%%%%%%%%%%%%%%%%%%%%%%%%%%%%%%%%%%%%%%%%%%%%%%%
% SETUP
%%%%%%%%%%%%%%%%%%%%%%%%%%%%%%%%%%%%%%%%%%%%%%%%

\begin{document}

% default font family
\sffamily

% spacing between words
\fontdimen2\font=0.2em

% set length of delimiter
\newlength{\DelimitlineLength}
\setlength{\DelimitlineLength}{\textwidth+1em}

% colors
\pagecolor[RGB]{245,245,245}

%%%%%%%%%%%%%%%%%%%%%%%%%%%%%%%%%%%%%%%%%%%%%%%%
% HEAD
%%%%%%%%%%%%%%%%%%%%%%%%%%%%%%%%%%%%%%%%%%%%%%%%

{\setlength\multicolsep{0pt}\begin{multicols}{2}
    \begin{spacing}{1.6}\raggedright\rmfamily
    {\huge \vphantom{Name: }Ilya\,Lesikov}\\[1.8ex]
    {\Large \vphantom{Title: }DevOps Engineer/SRE}\\
    {\Large \vphantom{Location: }Russia, Ryazan (UTC+3)}\\
    {\Large \vphantom{Age: }27 years old}
  \end{spacing}

  \columnbreak

  \begin{flushright}\begin{spacing}{1.3}\rmfamily
    \textit{Updated: 04/03/2020}\\[0.5ex]
    \large{
      \vphantom{contacts, contact information: }
      \faEnvelope \href{mailto:hire@lesikov.com}{\vphantom{Email: }\ hire@lesikov.com}\\
      \faInternetExplorer \href{https://ilya-lesikov.com}{\vphantom{Website: }\ ilya-lesikov.com}\\
      \faLinkedin \href{https://www.linkedin.com/in/ilya-lesikov}{\vphantom{LinkedIn: }\ ilya-lesikov}\\
      \faGithubAlt \href{https://github.com/ilya-lesikov}{\vphantom{GitHub: }\ ilya-lesikov}}
  \end{spacing}\end{flushright}
\end{multicols}}

\vspace{2ex}

%%%%%%%%%%%%%%%%%%%%%%%%%%%%%%%%%%%%%%%%%%%%%%%%
% ABOUT
%%%%%%%%%%%%%%%%%%%%%%%%%%%%%%%%%%%%%%%%%%%%%%%%

\subsection*{About\vphantom{ (professional summary)}}
\Delimitline

\forceindent Hey, I'm self-taught and my real-world experience may be not that impressive, but don't be so skeptical — I'm better than you might think and I'll pass that technical interview :) \href{https://github.com/ilya-lesikov/gke-demo}{This CI/CD/cloud automation demo} is an indicator of my current level, you can show this to the technical people.

\forceindent I solve problems and make things easy (not simple). No reinventing the wheel. No overcomplicating. I know the balance between doing it right and quickly. Generally "doing it right > quickly" though. My code is clean, maintainable and extensible, autotested and documented. I have more of a developer mindset, still a seasoned Linux SysAdmin.

\forceindent I'm a good man. Easy to work with. Strong, balanced, informed. Funny (lie). I never lie.

\forceindent Speaking English (Advanced/C1+) and Russian (native).

%%%%%%%%%%%%%%%%%%%%%%%%%%%%%%%%%%%%%%%%%%%%%%%%
% GITHUB
%%%%%%%%%%%%%%%%%%%%%%%%%%%%%%%%%%%%%%%%%%%%%%%%

\subsection*{GitHub}
\Delimitline

\forceindent To learn and to showcase my skills I recently did a demo of a pretty complete, fully-featured CI/CD and cloud automation for microservices. Done with Terraform, GCP, GKE/Kubernetes, GCB, ArgoCD, Stackdriver:\vspace{0.2em}\\
\rurl{github.com/ilya-lesikov/gke-demo}

Some Ansible/Docker automation with E2E and role-level testing:\vspace{0.2em}\\
\rurl{github.com/ilya-lesikov/ansible-personal-infra}

Also, things I did when I was looking for a job as a junior in 2017 (not an indication of my current qualifications):\vspace{0.2em}\\
\rurl{github.com/ilya-lesikov/permfixd}\\
\rurl{github.com/ilya-lesikov/CI-CD-pipeline}\\
\rurl{github.com/ilya-lesikov/taskmngr}

%%%%%%%%%%%%%%%%%%%%%%%%%%%%%%%%%%%%%%%%%%%%%%%%
% EXPERIENCE
%%%%%%%%%%%%%%%%%%%%%%%%%%%%%%%%%%%%%%%%%%%%%%%%

\subsection*{\vphantom{Professional Work }Experience}
\Delimitline

\begin{etaremune}[
  topsep=1ex,itemsep=1.5ex,partopsep=0ex,
  parsep=0ex,rightmargin=1em,leftmargin=2em
]
  \item
    \emph{Self education}\hfill
    \textit{6 mos.}\hspace{1.0em}
    \textit{07/19\,—\,01/20}\vspace{1.5ex}\newline
    \forceindent I improved my automation, security and networking expertise through books and practice. Made my second, more advanced cloud automation and CI/CD demo (\href{https://github.com/ilya-lesikov/gke-demo}{github.com/ilya-lesikov/gke-demo})

  \item
    \emph{SRE/DevOps Engineer}\hfill
    in Smart Transport\hspace{1.0em}
    \textit{12 mos.}\hspace{1.0em}
    \textit{12/17\,—\,12/18}\vspace{1.5ex}\newline
    \forceindent I did a fully-featured cloud automation and CI/CD done with Terraform, Jenkins, Chef, Docker. It was developed in parallel with the existing (and broken in many ways) infrastructure, so the actual migration of the applications still had to be finished. Migrated from broken Zabbix to Prometheus, eliminating a lot of the technical debt. 50\% of my work was maintaining and operating the existing infrastructure.\\[0.7ex]
    \forceindent Why left: no growth.

\newpage

  \item
    \emph{Self education}\hfill
    \textit{10 mos.}\hspace{1.0em}
    \textit{01/17\,—\,11/17}\vspace{1.5ex}\newline
    \forceindent Decided to learn how to develop software and do CI/CD the proper way. Ended up with my first CI/CD demo (Jenkins, Ansible, K8s), command-line app in C++ and web app in Python/Django (links in the end of the ``GitHub" section)

  \item
    \emph{System Administrator (Linux/Windows)}\hfill
    in \href{https://ivideon.com}{Ivideon}\hspace{1.0em}
    \textit{9 mos.}\hspace{1.0em}
    \textit{10/15\,—\,07/16}\vspace{1.5ex}\newline
    \forceindent My first real-world experience with automation. Partially automated machines provisioning and maintenance with Chef, Fabric and some ad-hoc Bash/PS scripting. It was ugly :)\\[0.7ex]
    \forceindent Why left: overqualified.

  \item
    \emph{Technical Support (English/Russian)}\hfill
    in \href{https://ivideon.com}{Ivideon}\hspace{1.0em}
    \textit{17 mos.}\hspace{1.0em}
    \textit{05/14\,—\,10/15}
\end{etaremune}

% somehow this decreases the padding after etaremune lists
\vspace{0ex}

%%%%%%%%%%%%%%%%%%%%%%%%%%%%%%%%%%%%%%%%%%%%%%%%
% Worked with
%%%%%%%%%%%%%%%%%%%%%%%%%%%%%%%%%%%%%%%%%%%%%%%%

\subsection*{Worked with\vphantom{ (skills)}}
\Delimitline

\forceindent Many things I won't mention (e.g. web servers, load balancers) since it would be too much and is not really important:
\begin{itemize}
  \item Cloud Platforms: Google Cloud Platform
  \item Cloud Automation: Terraform
  \item Container Orchestration: Kubernetes
  \item CI/CD: Jenkins, Google Cloud Build, ArgoCD
  \item CMS: Ansible, Chef
  \item Monitoring: Prometheus, Grafana, Zabbix
  \item Languages: mostly Bash and Python, but I worked some with a dozen different languages over time
  \item Few more keywords: Linux, Docker, Git, GKE, KVM, Ruby
\end{itemize}

%%%%%%%%%%%%%%%%%%%%%%%%%%%%%%%%%%%%%%%%%%%%%%%%
% Education
%%%%%%%%%%%%%%%%%%%%%%%%%%%%%%%%%%%%%%%%%%%%%%%%

\subsection*{\vphantom{Education: }}
\vphantom{ middle school}

\end{document}
