\documentclass[11pt, a4paper]{article}
\usepackage{geometry}
\geometry{a4paper, left=20mm, right=20mm, top=15mm, bottom=20mm}

\usepackage{fontspec}
\defaultfontfeatures{Renderer=Basic,Ligatures={TeX}, Numbers=Lining}
\setmainfont{Georgia}
\setsansfont{Carlito}
\setmonofont[Scale=0.95, LetterSpace=-1.5, FakeStretch=0.90]{SauceCodePro Nerd Font}
\usepackage[english,russian]{babel}

\usepackage[activate={true,nocompatibility},final,tracking=true,factor=1200,stretch=20,shrink=20]{microtype}

% section titles configuration
\usepackage{titlesec}
% colorize text (advanced)
\usepackage{xcolor}
% Enable multicolumns
\usepackage{multicol}
% multiline comments
\usepackage{verbatim}
% configure top/bottom padding for itemize lists
\usepackage{enumitem}

\usepackage{setspace}
%% reversed enumerated lists
%\usepackage{etaremune}

% configure sections
\usepackage{sectsty}
\usepackage[normalem]{ulem}

% PDF SETUP
\usepackage{hyperref}
\hypersetup {
  pdfauthor={Илья Лесиков},
  pdfsubject={Резюме (Devops)},
  pdftitle={Резюме (DevOps) — Илья Лесиков},
  colorlinks, breaklinks,
  filecolor=black,
  urlcolor=[RGB]{41,41,212},
  linkcolor=[rgb]{0.117,0.682,0.858},
  linkcolor=[rgb]{0.117,0.682,0.858},
  citecolor=[rgb]{0.117,0.682,0.858}
}

% CUSTOM COMMANDS
\newcommand{\Delimitline}{
  %\vspace{-5ex}
  {\rule{\linewidth}{0.13ex}}
}

\newcommand\rurl[1]{%
  \href{http://#1}{\nolinkurl{#1}}%
}

\newcommand\Eng[1]{%
  \foreignlanguage{english}{#1}%
}

%% HACK to decrease boldness globally
%\renewcommand{\textbf}[1]{%
%    \pdfliteral direct {2 Tr 0.2 w} %the second factor is the boldness 
%     #1%
%    \pdfliteral direct {0 Tr 0 w}%
%}
\let\bf\textbf

\let\it\textit

%% FORMATTING

% \sectionfont{\rmfamily\mdseries}
\subsectionfont{\rmfamily\mdseries\scshape\large}
%\subsubsectionfont{\rmfamily\mdseries\normalsize}

%\titlespacing*{\section}
%{0pt}{5.5ex plus 1ex minus .2ex}{4.3ex plus .2ex}
\titlespacing*{\subsection}{20pt}{20pt}{20pt}

\setlength{\parskip}{0.8ex}
% Do not indent paragraphs
\setlength\parindent{0in}
% suppress page numbers
\pagenumbering{gobble}

% border between tabular cells
\setlength{\arrayrulewidth}{0pt}
% paddings between columns in tabular
\setlength{\tabcolsep}{0em}

% itemize lists
\setlist[itemize]{itemsep=0ex, topsep=0ex}

% bulletpoints
\renewcommand\labelitemi{\textbullet}
\renewcommand\labelitemii{\footnotesize\raisebox{0.05ex}{\textbullet}}

%% DOCUMENT
\begin{document}

% default font family
\sffamily

%% COLORS
\pagecolor[RGB]{245,245,245}

{\setlength\multicolsep{0pt}%
\begin{multicols}{2}

\begin{spacing}{1.5}
  {\LARGE Илья\,Лесиков}\\
  {\Large\Eng{DevOps}}\hspace{1.3cm}{\large 25\,лет}
\end{spacing}

\columnbreak

\begin{flushright}
  \begin{spacing}{1.3}
    {\large\href{mailto:bititanc@gmail.com}{\Eng{bititanc@gmail.com}}}\\
    {\fontsize{1.4em}{0}\selectfont 8\,(930)\,88-66-248} \normalsize\\
  \end{spacing}
\end{flushright}

\end{multicols}
}

%\vspace{-10pt}
%\textcolor[RGB]{220,220,220}{\rule{\linewidth}{0.2pt}}
%\vspace{5pt}

\vspace{3ex}

\begin{comment}
У меня нет образования (неоконченный колледж электроники) и сомнительный стаж (10 месяцев сисадмином).\\
Но, самообучаясь, за последние пару лет я получил достаточный опыт и в общей автоматизации, и в \Eng{CI/CD}.\\
Получил некоторый опыт и в кодинге, как и хороший общий технический бэкграунд (\Eng{Linux}, сети и т.\,п.).
\end{comment}

%\vspace{-4ex}

\subsection*{Главное}
%\Delimitline
Демонстрация полностью автоматизированного развертывания \Eng{CI/CD} и приложения в нём на локальной машине 
%% HACK: shrink big spaces between words caused by boldness hack
%{\fontdimen2\font=1.5pt
(с \Eng{Vagrant, Ansible, Jenkins, Kubernetes, Zabbix, ELK}):\\
%\par
%}
\rurl{bititanb.github.io/CI-CD-pipeline}\\
\Delimitline

%\vspace{-4ex}

\subsection*{{GitHub}}
\Delimitline

Небольшой демон на \Eng{C++} (1100 строк + 600 строк тестов):\\
  \rurl{github.com/bititanb/permfixd}
  
Скрипты:
\begin{itemize}
  \item \Eng{Python}:\\
    \rurl{bitbucket.org/bititanb/linux-utils/src/master/xenserver-snapshot-revert.py}\\
    \rurl{bitbucket.org/bititanb/linux-utils/src/master/rpmbuild.py}\\
    \rurl{bitbucket.org/bititanb/linux-utils/src/master/pulseaudio-move-sink.py}
  \item \Eng{Bash}:\\
    \rurl{bitbucket.org/bititanb/linux-utils/src/master/alarm.sh}\\
    \rurl{bitbucket.org/bititanb/ivd-update-mango/src/master/update-mango.sh}
  \item \Eng{JavaScript}:\\
    \rurl{github.com/bititanb/google-snippets/blob/master/userscript.js}
  \item больше здесь:\\
    \rurl{bitbucket.org/bititanb/linux-utils/src}\\
    \rurl{bitbucket.org/bititanb/}\\
    \rurl{github.com/bititanb?tab=repositories&q=&type=source}
\end{itemize}
Небольшое приложение на \Eng{Django}, использовалось для \href{https://github.com/bititanb/CI-CD-pipeline}{\Eng{CI/CD} демо}:\\
  \rurl{github.com/bititanb/taskmngr}

\subsection*{{Основные технологии}}
\Delimitline
\begin{itemize}
  \item \href{https://github.com/bititanb/ansible-taskmngr}{Ansible}, \href{https://bitbucket.org/bititanb/ivd-fabric/src}{Fabric}, Jenkins, \href{https://github.com/bititanb/ansible-taskmngr/tree/master/roles/taskmngr-kubernetes/templates}{Kubernetes}, KVM/Xen/Docker, git
  \item \Eng{Linux} (отлично), \Eng{Windows}, сети
  \item \Eng{Bash, Python}, чуть \Eng{C++/C} и веб-разработки
  \begin{otherlanguage*}{english}
  \end{otherlanguage*}
\end{itemize}

Второстепенное ПО вроде \Eng{Vagrant}, сетевые и системные сервисы не перечисляю.\\
Ещё некоторое ПО \Eng{(Zabbix, ELK, Chef, PowerShell, \dots)} тоже не включил, т.\,к. хоть и использовал, но недолго.
% Тем не менее, старался поработать хотя бы с одним инструментом из каждой категории (CM, CI, мониторинг и т.\,д.), поэтому быстро разберусь в любом стеке.

\subsection*{{Опыт}}
\Delimitline
\begin{frame}

% TODO: do without itemize lists (no need for them)
\begin{spacing}{0.3}
\begin{tabular}{p{0.50\textwidth}p{0.15\textwidth}p{0.15\textwidth}p{0.20\textwidth}}

  \begin{itemize}
      \item[3.] Самообучение, работа над \href{https://github.com/bititanb/CI-CD-pipeline}{\Eng{CI/CD} демо}
  \end{itemize} &

  &
  
  \begin{itemize}
  \item[] 
    \it{10 мес.}
  \end{itemize} &

  \begin{itemize}
  \item[] 
    \it{01.17\,--\,11.17}
  \end{itemize} \\
  
  \begin{itemize}
  \begin{spacing}{0.2}
    \item[2.] Системный администратор \newline
  \end{spacing}
      \-\hspace{0.8em}\it{40-50 машин, \Eng{Linux/Windows} 50/50}
  \end{itemize} &
  
  \begin{itemize}
  \item[] 
  \begin{flushright}
    в \href{https://ivideon.com}{Ivideon}
  \end{flushright}
  \end{itemize} &

  \begin{itemize}
  \item[] 
    \it{10 мес.}
  \end{itemize} &

  \begin{itemize}
  \item[] 
    \it{10.15\,--\,07.16}
  \end{itemize} \\
  
  \begin{itemize}
      \item[1.] Англоязычная техподдержка 
  \end{itemize} &
  
  \begin{itemize}
  \item[] 
  \begin{flushright}
    в \href{https://ivideon.com}{Ivideon}
  \end{flushright}
  \end{itemize} &

  \begin{itemize}
  \item[] 
    \it{18 мес.}
  \end{itemize} &

  \begin{itemize}
  \item[] 
    \it{05.14\,--\,10.15}
  \end{itemize} \\

\end{tabular}
\end{spacing}

\end{frame}

% \begin{etaremune}
%   \item 
% {\setlength\multicolsep{0pt}%
% \begin{multicols}{2}
%   Самообучение, работа над \href{https://github.com/bititanb/CI-CD-pipeline}{\Eng{CI/CD} демо}
%   \columnbreak
%   \begin{flushright}
%   test1
%   \end{flushright}
%   \end{multicols}
%   }
%   \item Системный администратор (40-50 машин, \Eng{Linux/Windows} 50/50)
%   \item Англоязычная техподдержка
% \end{etaremune}

\subsection*{{Прочее}}
\Delimitline
\begin{itemize}
  \item Разговорный английский
  \item Рекомендации с прошлой работы
  \item Военный билет
\end{itemize}

\end{document}
