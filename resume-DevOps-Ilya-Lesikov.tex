\documentclass[11pt, a4paper]{article}
\usepackage{geometry}
\geometry{a4paper, left=15mm, right=15mm, top=15mm, bottom=20mm}

% Enable multicolumns
\usepackage{multicol}

\usepackage{setspace}
\usepackage{etaremune}
\usepackage{stackengine,tikz}

% LANG, FONTS
\usepackage[T2A]{fontenc}
\usepackage[utf8]{inputenc}
\usepackage[russian,english]{babel}

\usepackage[babel=true,activate={true,nocompatibility},final,tracking=true,kerning=true,spacing=true,factor=1100,stretch=10,shrink=10]{microtype}
% good fonts: gentium, droid
\usepackage{gentium}

% HEADINGS
% \usepackage{sectsty}
% \usepackage[normalem]{ulem}
% \sectionfont{\rmfamily\mdseries}
% \subsectionfont{\rmfamily\mdseries\scshape\normalsize}
% \subsubsectionfont{\rmfamily\bfseries\upshape\normalsize}

% PDF SETUP
\usepackage{hyperref}
\hypersetup
{
  pdfauthor={Илья Лесиков},
  pdfsubject={Резюме (Devops)},
  pdftitle={Резюме (DevOps) — Илья Лесиков},
  colorlinks, breaklinks,
  filecolor=black,
  urlcolor=[rgb]{0.117,0.682,0.858},
  linkcolor=[rgb]{0.117,0.682,0.858},
  linkcolor=[rgb]{0.117,0.682,0.858},
  citecolor=[rgb]{0.117,0.682,0.858}
}

% FORMATTING
\setlength{\parskip}{1em}
% Do not indent paragraphs
\setlength\parindent{0in}
% suppress page numbers
\pagenumbering{gobble}

% bulletpoints
\renewcommand\labelitemi{\textbullet}
\renewcommand\labelitemii{\footnotesize\raisebox{0.05ex}{\textbullet}}

% CUSTOM COMMANDS
\newcommand{\Delimitline}{
  \vspace{-4ex}
  \textcolor[RGB]{120,120,120}{\rule{\linewidth}{1pt}}
  \vspace{-4ex}
}

\newcommand\rurl[1]{%
  \href{http://#1}{\nolinkurl{#1}}%
}

\begin{document}

{\setlength\multicolsep{0pt}%
\begin{multicols}{2}

\begin{spacing}{1.5}
  {\LARGE Илья\,Лесиков}\\
  {\large DevOps}\hspace{1.3cm}{\large 25\,лет}
\end{spacing}

\columnbreak

\begin{flushright}
  \begin{spacing}{1.3}
    {\large \href{mailto:bititanc@gmail.com}{bititanc@gmail.com}}\\
    \fontsize{1.4em}{0}\selectfont 8\,(930)\,88-66-248 \normalsize\\
  \end{spacing}
\end{flushright}

\end{multicols}
}

\vspace{-10pt}
\textcolor[RGB]{220,220,220}{\rule{\linewidth}{0.2pt}}
\vspace{5pt}

У меня нет образования (неоконченный колледж электроники) и сомнительный стаж (10 месяцев сисадмином).\\
Но, самообучаясь, за последние пару лет я получил достаточный опыт и в общей автоматизации, и в CI/CD.\\
Получил некоторый опыт и в кодинге, как и хороший общий технический бэкграунд (Linux, сети и т.\,п.).

Как подтверждение навыков — демонстрация полностью автоматизированного развертывания CI/CD и приложения в нём на локальной машине (с Vagrant, Ansible, Jenkins, Kubernetes, Zabbix, ELK):

\rurl{bititanb.github.io/CI-CD-pipeline}

\subsection*{Дополнительно:}
\begin{itemize}
  \item Демон на C++ (1100 строк + 600 строк тестов):\\
    \rurl{github.com/bititanb/permfixd}
  \item Скрипты:
    \begin{itemize}
      \item Python:\\
        \rurl{bitbucket.org/bititanb/linux-utils/src/master/xenserver-snapshot-revert.py}\\
        \rurl{bitbucket.org/bititanb/linux-utils/src/master/rpmbuild.py}\\
        \rurl{bitbucket.org/bititanb/linux-utils/src/master/pulseaudio-move-sink.py}
      \item Bash:\\
        \rurl{bitbucket.org/bititanb/linux-utils/src/master/alarm.sh}\\
        \rurl{bitbucket.org/bititanb/ivd-update-mango/src/master/update-mango.sh}
      \item JavaScript:\\
        \rurl{github.com/bititanb/google-snippets/blob/master/userscript.js}
      \item больше здесь:\\
        \rurl{bitbucket.org/bititanb/linux-utils/src}\\
        \rurl{bitbucket.org/bititanb/}\\
        \rurl{github.com/bititanb?tab=repositories&q=&type=source}
    \end{itemize}
  \item Небольшое приложение на Django, использовалось для \href{https://github.com/bititanb/CI-CD-pipeline}{CI-CD демо}:\\
    \rurl{github.com/bititanb/taskmngr}
\end{itemize}

\subsection*{Основные технологии:}
\Delimitline
\begin{itemize}
  \item Linux (deb, rpm, \dots) и сети — очень хорошо, Windows
  \item sh/Bash, Python, чуть C++/C и JavaScript
  \item \href{https://github.com/bititanb/ansible-taskmngr}{Ansible}, \href{https://bitbucket.org/bititanb/ivd-fabric/src}{Fabric}, Jenkins, \href{https://github.com/bititanb/ansible-taskmngr/tree/master/roles/taskmngr-kubernetes/templates}{Kubernetes}, KVM, Xen, Docker, git\par
\end{itemize}

Веб-серверы, сетевые и системные сервисы, второстепенное ПО вроде Vagrant не перечисляю — там всё просто. Ещё некоторое ПО (Zabbix, ELK, Chef, \dots) тоже, т.\,к. хоть и использовал, но недостаточно плотно.
% Тем не менее, старался поработать хотя бы с одним инструментом из каждой категории (CM, CI, мониторинг и т.\,д.), поэтому быстро разберусь в любом стеке.

\subsection*{Прочее:}
\Delimitline
\begin{itemize}
  \item Разговорный английский
  \item Военный билет
\end{itemize}

\subsection*{Опыт работы}
\Delimitline
\begin{etaremune}
  \item Самообучение, работа над \href{https://github.com/bititanb/CI-CD-pipeline}{CI-CD демо}
  \item Системный администратор (40-50 машин, Linux/Windows 50/50)
  \item Англоязычная техподдержка
\end{etaremune}

\end{document}
