%!TEX TS-program = xelatex
%!TEX encoding = UTF-8 Unicode

\documentclass[10pt, a4paper]{article}

% LAYOUT
%--------------------------------
% Margins
\usepackage{geometry}
\geometry{a4paper, left=20mm, right=20mm, top=15mm, bottom=20mm}

% Do not indent paragraphs
\setlength\parindent{0in}

% Enable multicolumns
\usepackage{multicol}
% \setlength{\columnsep}{-3.5cm}
% \setlength{\columnsep}{}

% Uncomment to suppress page numbers
\pagenumbering{gobble}

% LANGUAGE
%--------------------------------
% Set the main language
\usepackage{polyglossia}
\setmainlanguage{ru}

% TYPOGRAPHY
%--------------------------------
\usepackage{fontspec}
\usepackage{xunicode}
\usepackage{xltxtra}
% converts LaTeX specials (quotes, dashes etc.) to Unicode
\defaultfontfeatures{LetterSpace=2, Mapping=tex-text}
\setromanfont [Ligatures={Common}, Numbers={OldStyle}]{Carlito}
% Cool ampersand
\newcommand{\amper}{{\fontspec[Scale=.95]{Carlito}\selectfont\itshape\&}}

% MARGIN NOTES
%--------------------------------
% \usepackage{marginnote}
% \newcommand{\note}[1]{\marginnote{\scriptsize #1}}
% \renewcommand*{\raggedleftmarginnote}{}
% \setlength{\marginparsep}{7pt}
% \reversemarginpar

% HEADINGS
%--------------------------------
\usepackage{sectsty}
\usepackage[normalem]{ulem}
\sectionfont{\rmfamily\mdseries}
\subsectionfont{\rmfamily\mdseries\scshape\normalsize}
\subsubsectionfont{\rmfamily\bfseries\upshape\normalsize}

% PDF SETUP
%--------------------------------
\usepackage{hyperref}
\hypersetup
{
  pdfauthor={Илья Лесиков},
  pdfsubject={Резюме (Devops)},
  pdftitle={Резюме (DevOps) — Илья Лесиков},
  colorlinks, breaklinks, xetex, bookmarks,
  filecolor=black,
  urlcolor=[rgb]{0.117,0.682,0.858},
  linkcolor=[rgb]{0.117,0.682,0.858},
  linkcolor=[rgb]{0.117,0.682,0.858},
  citecolor=[rgb]{0.117,0.682,0.858}
}

% CUSTOM CONFIG
\usepackage{setspace}

\usepackage{stackengine,tikz}

\usepackage{microtype}

\setlength{\parskip}{1em}

\renewcommand\labelitemi{\textbullet}
\renewcommand\labelitemii{\footnotesize\raisebox{0.05ex}{\textbullet}}

\newcommand{\Delimitline}{
  \vspace{-4ex}
  \textcolor[RGB]{120,120,120}{\rule{\linewidth}{1pt}}
  \vspace{-4ex}
}

\usepackage{etaremune}

% DOCUMENT
%--------------------------------
% \renewcommand{\baselinestretch}{.1}
% \setlength{\parindent}{4em}
% \setlength{\parskip}{4em}
% \onehalfspacing
\begin{document}

% {\setstretch{0.1} $name$}\\
% \begin{spacing}{0.3}
%   {\LARGE Илья Лесиков}\\
% \end{spacing}

% \vspace{-20pt}

{\setlength\multicolsep{0pt}%
\begin{multicols}{2}

% $for(address)$
% $address$\\
% $endfor$

% \vspace{-10pt}

% $for(urls)$
% \href{http://$urls$}{$urls$}\\
% $endfor$

% \begin{spacing}{1}
%   \-\hspace{0.3cm}{\large 25 лет}\\
% \end{spacing}

\begin{spacing}{1.5}
  % \-\hspace{0.3cm}
  {\LARGE Илья\,Лесиков}\\
  {\large DevOps}\hspace{1.3cm}{\large 25\,лет}\\
\end{spacing}

\columnbreak

\begin{flushright}
  \begin{spacing}{1.3}
    {\large \href{mailto:bititanc@gmail.com}{bititanc@gmail.com}}\\
    \fontsize{1.4em}{0}\selectfont 8\,(930)\,88-66-248 \normalsize\\
  \end{spacing}
\end{flushright}

\end{multicols}
}

\vspace{-10pt}
\textcolor[RGB]{220,220,220}{\rule{\linewidth}{0.2pt}}
\vspace{5pt}

У меня нет образования (неоконченный колледж электроники) и сомнительный стаж (8 месяцев сисадмином).\\
Но, самообучаясь, за последние пару лет я получил достаточный опыт и в общей автоматизации, и в CI/CD.\\
Получил некоторый опыт и в кодинге, как и хороший общий технический бэкграунд (Linux, сети и т.\,п.).

Как подтверждение навыков — демонстрация полностью автоматизированного развертывания CI/CD и приложения в нём на локальной машине (с Vagrant, Ansible, Jenkins, Kubernetes, Zabbix, ELK):

\url{https://bititanb.github.io/CI-CD-pipeline}

\subsection*{Дополнительно:}
\begin{itemize}
  \item Демон на C++ (1100 строк + 600 строк тестов):\\
    \url{https://github.com/bititanb/permfixd}
  \item Скрипты:
    \begin{itemize}
      \item Python:\\
        \url{https://bitbucket.org/bititanb/linux-utils/src/master/xenserver-snapshot-revert.py}\\
        \url{https://bitbucket.org/bititanb/linux-utils/src/master/rpmbuild.py}\\
        \url{https://bitbucket.org/bititanb/linux-utils/src/master/pulseaudio-move-sink.py}
      \item Bash:\\
        \url{https://bitbucket.org/bititanb/linux-utils/src/master/alarm.sh}\\
        \url{https://bitbucket.org/bititanb/ivd-update-mango/src/master/update-mango.sh}
      \item JavaScript:\\
        \url{https://github.com/bititanb/google-snippets/blob/master/userscript.js}
      \item больше здесь:\\
        \url{https://bitbucket.org/bititanb/linux-utils/src}\\
        \url{https://bitbucket.org/bititanb/}\\
        \url{https://github.com/bititanb?tab=repositories&q=&type=source}
    \end{itemize}
  \item Небольшое приложение на Django, использовалось для \href{https://github.com/bititanb/CI-CD-pipeline}{CI-CD демо}:\\
    \url{https://github.com/bititanb/taskmngr}
\end{itemize}

\subsection*{Основные технологии:}
\Delimitline
% \vspace{-4ex}
% \begin{spacing}{0.8}
%   \begin{flushright}
%     \textcolor[RGB]{120,120,120}{\rule{\linewidth}{1pt}}\\
%     \textit{\footnotesize Веб-серверы, сетевые и системные сервисы, второстепенное ПО вроде Vagrant не перечисляю — там всё просто}
%   \end{flushright}
% \end{spacing}
% \vspace{-3ex}
\begin{itemize}
  \item Linux (deb, rpm, \dots) и сети — очень хорошо, Windows
  \item sh/Bash, Python, чуть C++/C и JavaScript
  \item \href{https://github.com/bititanb/ansible-taskmngr}{Ansible}, \href{https://bitbucket.org/bititanb/ivd-fabric/src}{Fabric}, Jenkins, \href{https://github.com/bititanb/ansible-taskmngr/tree/master/roles/taskmngr-kubernetes/templates}{Kubernetes}, KVM, Xen, Docker, git\par
\end{itemize}

Веб-серверы, сетевые и системные сервисы, второстепенное ПО вроде Vagrant не перечисляю — там всё просто. Ещё некоторое ПО (Zabbix, ELK, Chef, \dots) тоже, т.\,к. хоть и использовал, но недостаточно плотно.
% Тем не менее, старался поработать хотя бы с одним инструментом из каждой категории (CM, CI, мониторинг и т.\,д.), поэтому быстро разберусь в любом стеке.

\subsection*{Прочее:}
\Delimitline
\begin{itemize}
  \item Разговорный английский
  \item Военный билет
\end{itemize}

\subsection*{Опыт работы}
\Delimitline
\begin{etaremune}
  \item Самообучение, отдых, работа над \href{https://github.com/bititanb/CI-CD-pipeline}{CI-CD демо}
  \item Системный администратор (40-50 машин, Linux/Windows 50/50)
  \item Англоязычная техподдержка
\end{etaremune}

% \subsection*{Areas of Interest}
% $if(skills)$
% \begin{itemize}
%     $for(skills)$
%       \item $skills$
%     $endfor$
% \end{itemize}
% $endif$

% % \vfill

% \vspace{25pt}

% \section*{Previous Experience}
% \noindent
% $for(experience)$
% % \note{$experience.years$}\textsc{$experience.employer$}\\
% \emph{$experience.job$}\\
% $experience.city$\\[.2cm]
% $endfor$

% $if(education)$
% \section*{Education}
% \noindent
% $for(education)$
% % \note{$education.year$}\textbf{$education.subject$}$if(education.degree)$, $education.degree$$endif$\\
% \emph{$education.institute$}$if(education.city)$, $education.city$$endif$\\[.2cm]
% $endfor$
% $endif$

% $if(languages)$
% \section*{Languages}
% $for(languages)$
% \emph{$languages.language$} ($languages.proficiency$)\\
% $endfor$
% $endif$

\end{document}
