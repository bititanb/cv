\documentclass[11pt, a4paper]{article}
\usepackage{geometry}
\geometry{a4paper, left=25mm, right=25mm, top=10mm, bottom=5mm}

\usepackage{fontspec}
\defaultfontfeatures{Renderer=Basic,Ligatures={TeX}, Numbers=Lining}
\setmainfont{Georgia}
\setsansfont[FakeStretch=1.05]{Carlito}
\setmonofont[Scale=0.95, LetterSpace=-1.5, FakeStretch=0.90]{SauceCodePro Nerd Font}
\usepackage[english,russian]{babel}

\usepackage[activate={true,nocompatibility},final,tracking=true,factor=1200,stretch=20,shrink=20]{microtype}

% disable hyphenation
\usepackage[none]{hyphenat}
% arithmetic operations in latex
\usepackage{calc}
% section titles configuration
\usepackage{titlesec}
% colorize text (advanced)
\usepackage{xcolor}
% Enable multicolumns
\usepackage{multicol}
% multiline comments
\usepackage{verbatim}
% configure top/bottom padding for itemize lists
\usepackage{enumitem}

\usepackage{setspace}
%% reversed enumerated lists
\usepackage{etaremune}

% PDF SETUP
\usepackage{hyperref}
\hypersetup {
  pdfauthor={Илья Лесиков},
  pdfsubject={Резюме (Devops)},
  pdftitle={Резюме (DevOps) — Илья Лесиков},
  colorlinks, breaklinks,
  filecolor=black,
  urlcolor=[RGB]{0,0,208},
  linkcolor=[RGB]{0,0,208},
  citecolor=[RGB]{0,0,208},
}

% CUSTOM COMMANDS
\newcommand{\Delimitline}{
  \vspace{-2ex}
  \noindent\makebox[\linewidth]{\rule{\DelimitlineLength}{0.12ex}} }

\newcommand\rurl[1]{%
  \-\hspace{0.5em}
  \href{http://#1}{\nolinkurl{#1}}%
}

\newcommand\Eng[1]{%
  \foreignlanguage{english}{#1}%
}

\let\bf\textbf

\let\it\textit

\newcommand{\forceindent}{\leavevmode{\parindent=1.5em\indent}}

%% FORMATTING

% \sectionfont{\rmfamily\mdseries}
%\subsectionfont{\rmfamily\mdseries\scshape\large}
%\subsubsectionfont{\rmfamily\mdseries\normalsize}

\titlespacing*{\subsection}{1.3em}{2ex}{0ex}
\titleformat{\subsection}{\rmfamily\mdseries\scshape\large}{\thesubsection}{0.7em}{}

\setlength{\parskip}{1ex}
% Do not indent paragraphs
\setlength{\parindent}{0ex}
% suppress page numbers
\pagenumbering{gobble}

% border between tabular cells
\setlength{\arrayrulewidth}{0pt}
% paddings between columns in tabular
\setlength{\tabcolsep}{0em}

% itemize lists
\setlist[itemize]{itemsep=0ex, topsep=0ex, leftmargin=2em}

% bulletpoints
\renewcommand\labelitemi{\textbullet}
%\renewcommand\labelitemii{\footnotesize\raisebox{0.05ex}{\textbullet}}

%% DOCUMENT
\begin{document}

% default font family
\sffamily

% spacing between words
\fontdimen2\font=0.2em

% maths
\newlength{\DelimitlineLength}
\setlength{\DelimitlineLength}{\textwidth+1em}

%% COLORS
\pagecolor[RGB]{245,245,245}

{\setlength\multicolsep{0pt}%
\begin{multicols}{2}

\begin{spacing}{1.5}
  \rmfamily
  {\LARGE Илья\,Лесиков}\\
  {\Large\Eng{DevOps}}\hspace{1cm}{\large 25\,лет}
\end{spacing}

\columnbreak

\begin{flushright}
  \begin{spacing}{1.5}
    \rmfamily
    {\large\href{mailto:bititanc@gmail.com}{\Eng{bititanc@gmail.com}}}\\
    {\fontsize{1.4em}{0}\selectfont 8\,(930)\,88-66-248} \normalsize\\
  \end{spacing}
\end{flushright}

\end{multicols}
}

\vspace{2ex}

\begin{comment}
У меня нет образования (неоконченный колледж электроники) и сомнительный стаж (10 месяцев сисадмином).\\
Но, самообучаясь, за последние пару лет я получил достаточный опыт и в общей автоматизации, и в \Eng{CI/CD}.\\
Получил некоторый опыт и в кодинге, как и хороший общий технический бэкграунд (\Eng{Linux}, сети и т.\,п.).
\end{comment}

\subsection*{GitHub}
\Delimitline

{
\raggedright
\forceindent Главное — демонстрация полностью автоматизированного развертывания \Eng{CI/CD} и приложения с ним на~локальной~машине
(с \Eng{Kubernetes, Ansible, Jenkins, ELK, Zabbix, Vagrant}):
\vspace{0.4ex}\\
  \rurl{bititanb.github.io/CI-CD-pipeline}
}
  
\vspace{2.5ex}

Небольшой демон на \Eng{C++} (1100 строк + 600 строк тестов):
\vspace{0.4ex}\\
  \rurl{github.com/bititanb/permfixd}
  
Скрипты:
\begin{itemize}
  \item \Eng{Python}:
    \vspace{0.2em}\\
    \rurl{bitbucket.org/bititanb/linux-utils/src/master/xenserver-snapshot-revert.py}\\
    \rurl{bitbucket.org/bititanb/linux-utils/src/master/rpmbuild.py}\\
    \rurl{bitbucket.org/bititanb/linux-utils/src/master/pulseaudio-move-sink.py}
  \item \Eng{Bash}:
    \vspace{0.2ex}\\
    \rurl{bitbucket.org/bititanb/linux-utils/src/master/alarm.sh}\\
    \rurl{bitbucket.org/bititanb/ivd-update-mango/src/master/update-mango.sh}
  \item \Eng{JavaScript}:
    \vspace{0.2ex}\\
    \rurl{github.com/bititanb/google-snippets/blob/master/userscript.js}
  \item больше здесь:
    \vspace{0.2ex}\\
    \rurl{bitbucket.org/bititanb/linux-utils/src}\\
    \rurl{bitbucket.org/bititanb/}\\
    \rurl{github.com/bititanb?tab=repositories&q=&type=source}
\end{itemize}

Небольшое приложение на \Eng{Django}, использовалось для \href{https://github.com/bititanb/CI-CD-pipeline}{\Eng{CI/CD} демо}:
\vspace{0.4ex}\\
  \rurl{github.com/bititanb/taskmngr}

\subsection*{{Основные технологии}}
\Delimitline

\begin{itemize}
  \item \href{https://github.com/bititanb/ansible-taskmngr}{Ansible}, \href{https://bitbucket.org/bititanb/ivd-fabric/src}{Fabric}, Jenkins, \href{https://github.com/bititanb/ansible-taskmngr/tree/master/roles/taskmngr-kubernetes/templates}{Kubernetes}, KVM/Xen/Docker, git
  \item \Eng{Linux} (отлично), \Eng{Windows}, сети
  \item \Eng{Bash, Python}, чуть \Eng{C++/C} и веб-разработки
  \begin{otherlanguage*}{english}
  \end{otherlanguage*}
\end{itemize}

\forceindent Второстепенное~ПО вроде~\Eng{Vagrant}, сетевые и системные сервисы не~перечисляю.
Ещё некоторое \Eng{(Zabbix,~ELK,~Chef,~PowerShell,~\dots)} тоже не~включил, т.\,к. хоть и использовал, но недолго.
% Тем не менее, старался поработать хотя бы с одним инструментом из каждой категории (CM, CI, мониторинг и т.\,д.), поэтому быстро разберусь в любом стеке.

\subsection*{{Опыт}}
\Delimitline

\begin{spacing}{0.8}
\begin{etaremune}[
  topsep=1ex,itemsep=1.5ex,partopsep=0ex,
  parsep=0ex,rightmargin=2em,leftmargin=2em
]
\item
    Самообучение, работа над \href{https://github.com/bititanb/CI-CD-pipeline}{\Eng{CI/CD} демо}
  \hfill 
    \it{10 мес.}
  \hskip 1.5em minus 1em 
    \it{01.17\,--\,11.17}
  
\item
    Системный администратор
  \hfill 
    в \href{https://ivideon.com}{Ivideon}
  \hskip 2em minus 1.5em 
    \it{10 мес.}
  \hskip 1.5em minus 1em 
    \it{10.15\,--\,07.16}
  \newline
  \-\hspace{1em}
    \it{40-50 машин, \Eng{Linux/Windows} 50/50}
  
\item
    Англоязычная техподдержка клиентов
  \hfill 
    в \href{https://ivideon.com}{Ivideon}
  \hskip 2em minus 1.5em 
    \it{18 мес.}
  \hskip 1.5em minus 1em 
    \it{05.14\,--\,10.15} 
  \newline
  \-\hspace{1em}
    \it{звонки, чат, заявки}
\end{etaremune}
\end{spacing}

% somehow this lowers the padding after etaremune lists
\vspace{0ex}

\subsection*{Прочее}
\Delimitline

\begin{itemize}
  \item Разговорный английский
  \item Рекомендации с прошлой работы
  \item Военный билет
\end{itemize}

\hfill
  \it{Актуально:
\hskip 0.3em 
  03.12.17}
\hskip 0.3em \-

\end{document}
